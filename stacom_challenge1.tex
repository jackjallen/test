\documentclass{llncs}

\usepackage{amsmath}
\usepackage{graphicx}
\usepackage{epstopdf}
\usepackage{subfigure}
\usepackage{float}

\graphicspath{{./figures/}}
\begin{document}

\title{Myocardial infarction detection from Left Ventricular shapes using a Random Forest.}

\author{*}

\institute
{*\\
\email{*}\\
}
\maketitle              % typeset the title of the contribution
\begin{abstract}
Understanding myocardial remodelling, and developing tools for its accurate quantification, is desirable to improve the diagnosis and treatment of myocardial infarction patients. Clinical conventional metrics such as blood pool volume or ejection fraction are not always distinctive. Here we describe a method for the classification of myocardial infarction from 3D diastolic and systolic left ventricle shapes, represented by point sets. Classification features included global geometric, shape and thickness descriptors, and a random forest was used for classification. Results from cross validation show an accuracy of 92.5\% (leave-one-out) and 91.5\% (5-fold), improving the 85\% obtained by ejection fraction. These results suggest that refined remodelling metrics provide information beyond standard clinical descriptors.

\keywords{statistical shape model, random forest, left ventricle, Myocardial infarction}
\end{abstract}
%
\section{Introduction}
The damage resulting from myocardial infarction can cause the heart to remodel, which can have a negative effect on its function \cite{Sutton2000}. Remodelling of the left ventricle (LV) is particularly significant because of its potential life-threatening effects, and is currently characterised through global metrics, such as blood pool volume or LV ejection fraction. In particular, myocardial infarction can cause thinning and a decrease in thickness change over the cardiac cycle, in the area of the scar, while hypertrophy can occur in other areas as they adapt to compensate \cite{Sutton2000}, and these are features not captured by current clinical metrics. 

 

Shape and cardiac remodelling can also be characterised through statistical shape models, also known as computational statistical atlases \cite{Cootes1992,Young2009}. Atlas-based metrics have the potential to increase  specificity and sensitivity, and to improve our understanding of shape changes in development and pathology \cite{Lewandowski2012,Lorenz2006}. In this work we explore this hypothesis, generating a large collection of LV shape metrics, and analysing its joint classification performance through Breiman's decision tree ensembles, known as \emph{random forests} \cite{Breiman2001}, a technique that has been shown to provide excellent classification results when many features are available. The objective is to provide metrics that capture the detailed anatomical changes caused by infarction.


\section{Materials and Methods}
	\subsection{The data}
A dataset of 400 3D left ventricle shapes was provided by the organisers of the MICCAI 2015 Statistical Atlases and Computational Modeling of the heart (STACOM) workshop. Each case included end diastolic (ED) and end systolic (ES) point coordinates (1089 points) and indices of triangle vertices previously fitted to magnetic resonance images as described in \cite{Young2000}, with a correction for potential bias from different acquisition protocols \cite{Medrano-Gracia2013}. For each of the cases, the index of a point approximately in the middle of the septum was also provided. Of the 400 shapes, 200 were labelled for use as training data. Of the training data, 100 were cases with myocardial infarction, from the DETERMINE trial \cite{Kadish2009}. The remaining 100 training shapes were from the MESA study \cite{Bild2002} and did not show symptoms of myocardial infarction (these two cohorts will be referred to as DETERMINE and MESA, respectively). The remaining 200 unlabelled cases were used as an evaluation cohort by the challenge organisers, for which the results were submitted as an entry to the challenge. 


	\subsection{Feature computation}
	We calculated a combination or standard clinical features and novel atlas-based metrics, as described next (full list in table \ref{table:featurelist}). All metrics were computed using \emph{\mbox{MATLAB}} and \emph{The Visualization Tookit} (VTK) \cite{Schroeder2006}.
	\subsubsection{Volumes and ejection fraction.}
We calculated the volumes contained by the epicardium and endocardium surfaces, both at ED and ES. For this purpose, we closed each surface by adding a node located at the centroid of the basal opening, which we connected to the nodes on the edge of this opening to form triangles. These volumes were used to compute ejection fractions. 
\subsubsection{Sphericity.}Remodeling has been shown to produced increased sphericity of the left ventricle \cite{Sutton2000}. We  computed the sphericity \cite{Wadell1933} of the endocardium and epicardium surfaces at ED and ES using Eq. \ref{eq:sphericity}, where V and A are the volume and surface area of a 3D shape, respectively. Surface areas were calculated by summing the areas of all the triangles in each mesh.

\begin{equation} \label{eq:sphericity}
 Sphericity = \cfrac{\pi^{1/3}(6V)^{2/3}}{A} 
\end{equation}

\subsubsection{Atlas-based shape metrics.} 

A statistical shape model represented by a point distribution model was built in a similar way to previous publications \cite{VanAssen2006,Lotjonen2004}. The 200 cases were used to build the model. Firstly, three iterations of Procrustes rigid alignment were applied to the training shapes, using the first shape in the list of DETERMINE shapes as the reference in the first iteration, and the mean of the aligned shapes $\bar{\textbf{x}} $ in subsequent iterations. Principal component analysis (PCA) was then applied to the training shapes, to find a matrix of eigenvectors $\Phi$, with corresponding eigenvalues describing geometric modes of variation across the shapes. After ranking the eigenvectors in descending order according to their corresponding eigenvalues, the first 100 were used in equation \ref{eq:shape_model}, to obtain atlas-based shape metrics $\textbf{b} $ for each case, each shape vector $ \textbf{x} $. 


\begin{equation} \label{eq:shape_model}
\textbf{b} = \Phi^T(\textbf{x} - \bar{\textbf{x}} )
\end{equation}

\subsubsection{Myocardium thicknesses.}
We calculated the myocardium thicknesses for all the cases, at ES and ED. This was done by computing the distance between each node on the epicardium mesh and the closest point (not necessarily a node) on the endocardium surface. The shapes were divided into three segments, named: apical, middle and basal. Each segment consisted of an equal number of nodes in the long axis direction. Thickness statistics were computed for the whole left ventricle and each of these sections.
\subsubsection{Atlas-based thickness metrics.} 

We also computed a statistical shape model of the thickness of the 100 training cases. Modes of thickness were found in a similar way to those of shape (i.e using equation \ref{eq:shape_model}), but by applying PCA only on the MESA shapes, with the aim of making the model independent from the location of the scar within the ventricle, as well as the variability of the remodelling responses.

	
	\subsection{Classification}

\label{method:classification}

 

Random forests are ensembles of many decision trees built iteratively. To produce each tree, a subset of the cases, of predefined size, is randomly chosen (\emph{i.e.} 'bagged') to be used for training. At an initial node, a subset of the features is randomly chosen and the feature that provides the optimal split is identified and used to divide the training data into two branches, to two new nodes. In our study, branches were produced until each new node only contained one class. Unused cases (\emph{i.e.} 'out-of-bag' cases), were passed down each completed tree, and the most common prediction for each case was compared with the true label.

Classification error was obtained by cross-validation, using 'k-fold' techniques with k = 5 and k = 200 (equivalent to a 'leave-one-out'). We used 300 trees in the ensemble, as this was sufficient for achieving a error convergence. The 'out-of-bag' importance of each feature for successful classification was found individually, by randomly mixing the feature values, before passing the 'out-of-bag' cases down the tree again. The importance of each feature was the average increase in classification error caused by the mixing, for the whole ensemble, divided by the the standard deviation.

The most relevant shape and thickness atlas metrics were selected accordingly to their ‘out-of-bag’ importance:he first 15 modes were chosen as features to build the final random forest. The features in Table \ref{table:featurelist} were also used to classify the test data as an entry to the challenge. The parameters for shape and thickness variation in the test shapes were produced by fitting the test shapes to the training data models.


\begin{table}

		\begin{center}
	
		\begin{tabular}{l|l|}
\textbf{Index} \space & \multicolumn{1}{l}{\textbf{Feature}}
\\
 \hline
\cline{1-2}
 1 & Ejection fraction \\
\cline{1-2}
 2:16 & Shape model parameters for modes 1:15\\
\cline{1-2}
 17:20 & Sphericity (Endocardium and Epicardium at diastole and systole)\\
 \cline{1-2}
 21:22 & Mean thickness (diastole and systole)\\
  23 & Absolute difference in mean thickness between diastole and systole\\
 
  24:25 & Mode thickness (diastole and systole)\\
  
  26:27 & Median thickness (diastole and systole)\\
  
  28:29 & Thickness variance (diastole and systole)\\
  
  30 & Thickness difference variance\\
  
  31 & Thickness variance (diastole with systole)\\
   \cline{1-2}
  32:34 & Segment thickness variance (diastole: apical, middle and basal)\\
   35:37 & Segment thickness variance (systole: apical, middle and basal)\\
  
  38:40 & Segment thickness variance (diastole with systole: apical, middle and basal)\\
  \cline{1-2}
  41:43 & Segment thickness mean (diastole: apical, middle and basal)\\
  44:46 & Segment thickness mean (systole: apical, middle and basal)\\
   47:49 & Segment thickness mean (diastole with systole: apical, middle and basal)\\
   \cline{1-2}
   50:51 & Volumes (systole Endocardium and Epicardium) \\
   52:53 & Volumes (diastole Endocardium and Epicardium) \\
   \cline{1-2}
    54:55 & Log volumes (systole Endocardium and Epicardium) \\
   56:57 & Log volumes (diastole Endocardium and Epicardium) \\
    \cline{1-2}
    58:72 & Thickness model parameters for modes 1:15\\

     \cline{1-2}
     \end{tabular}

		\caption{The complete list of features used for classification.}
		\label{table:featurelist}
		\end{center}
		\end{table}

\section{Results}
	\subsection{Classification}
Ejection fraction is used to define a baseline for the classification performance in a conventional clinical setting, reporting an accuracy of 87\% (provided by two different thresholds as seen in \ref{fig:EFs}). This result is improved to a 92.5\% using proposed classification strategy, as reported in Figure    \ref{fig:classification_error__5kfold_LOO__all_72_dia_sys}. 

This result is achieved after the selection of atlas-based features guided by the out-of-bag importance illustrated in Fig. \ref{fig:random_forest_feature_importance__100_shape_eigenvalues__and_100_thickness_modes_dia_sys} . Figure \ref{fig:random_forest_feature_importance__all_72__dia_sys} shows the importance of the complete list of features (described in Table \ref{table:featurelist}), and reveals that the relative importance of the atlas-based metrics has changed. The most relevant mode of shape variation is illustrated in Figure \ref{fig:visualise_shapes}, where a variation in thickness can be seen at the apex at ED, and changes in endocardium curvature, particularly opposite the middle of the septum, can be seen at ES. 

		
		\begin{figure}[]
		\qquad
\begin{center} 
\subfigure[]{\includegraphics[width = 0.45\textwidth]{Ejection_fraction_histograms.eps}
 \label{fig:Ejectionfractionhistograms}}

\!
\subfigure[]{\includegraphics[width = 0.45\textwidth ]{sensitivity_versus_specificity_EF___leave_one_out.eps} \label{fig:sensitivity_versus_specificity_EF}}

\caption{ \ref{fig:Ejectionfractionhistograms} Comparing the ejection fractions of the DETERMINE and MESA cases \ref{fig:sensitivity_versus_specificity_EF} Cross-validated (leave-one-out) ejection fraction specificity and sensitivity. Yellow stars mark the sensitivity and specificity values for the optimum accuracy of 87\%.}
\label{fig:EFs}
\end{center}    

		\end{figure}

		\begin{figure}[]
		\begin{center}
		
		\includegraphics[height = 0.34\textwidth]{100_mode_importances_shape_and_thickness.eps}
	\caption{Importance of the first 100 modes of variation of the two models.}	
\label{fig:random_forest_feature_importance__100_shape_eigenvalues__and_100_thickness_modes_dia_sys}			\end{center}
		\end{figure}

	
		\begin{figure}[]
		\begin{center}
		
		\includegraphics[height = 0.34\textwidth]{random_forest_feature_importance__all_72_line_plot__dia_sys.eps}
		
\caption{Random forest importance of all the features specified in Table \ref{table:featurelist}. The colours group the features according to Table \ref{table:featurelist}. (yellow = shape model parameters, magenta = global thickness statistics, cyan = segment thickness variances, red = segment thickness means, blue circles = volumes, blue triangles = log volumes, black = thickness model parameters).}
\label{fig:random_forest_feature_importance__all_72__dia_sys}		\end{center}
	\end{figure}
	
	\begin{figure}[]
	\begin{center}
	

		\includegraphics[height = 0.385\textwidth]{classification_error__5kfold_LOO__all_72_dia_sys.eps}
\caption{Cross-validated (5-fold and leave-one-out) classification error for the random forest built using all the features listed in Table \ref{table:featurelist}}.	
\label{fig:classification_error__5kfold_LOO__all_72_dia_sys}			\end{center}
		\end{figure}
		
	\begin{figure}[]	
	\!		
\begin{center}   
\subfigure[end-diastole mode 13: -3 $\mathrm{std}$, mean, +3$\mathrm{std}$.]
{
\includegraphics[width=0.13\textwidth]{diastole_mode13_std-3.png} 

\includegraphics[width=0.13\textwidth]
{diastole_mean.png}

\includegraphics[width=0.13\textwidth]
{diastole_mode13_std3.png} \label{fig:ED_shape_mode_vis}
}
\qquad
\subfigure[end-systole mode 13: -3$\mathrm{std}$, mean, +3$\mathrm{std}$.]{
\includegraphics[width=0.13\textwidth]
{systole_mode13_std-3.png}
\label{fig:ES_shape_mode_vis}

\includegraphics[width=0.13\textwidth]{systole_mean.png} 

\includegraphics[width=0.13\textwidth]
{systole_mode13_std3.png} 
}


\caption{ Variation for the statistical shape model mode with the highest importance when used with all other features. The points approximately in line with the middle of the septum are shown in black in. }
\label{fig:visualise_shapes}
\end{center}
		\end{figure}

\section{Discussion}
As expected accordingly to clinical practice, ejection fraction proved to be a robust marker (87\% accuracy). Additional features improved the classification between infarct and control subjects, reaching a 92.5\% accuracy.

In the search of the most relevant classification markers, our results reveal that ES endocardium volume is slightly more important than ejection fraction (Fig. \ref{fig:random_forest_feature_importance__all_72__dia_sys}). Despite both measures being strongly correlated, they still reveal a large complementary value. The third metric that contributed to the decisions in our classifier was the variance in the thickness difference between ED and ES, a surrogate of the heterogeneity of contraction (infarcted cases will have a reduced contraction in scarred areas). Relatively, the atlas-based metrics made a small contribution to the classification. The analysis revealed additional insights about the remodelling pattern in infarction, with changes in endocardial curvature and apical thickness detected.

When using only shape modes to build a random forest, figure     \ref{fig:random_forest_feature_importance__100_shape_eigenvalues__and_100_thickness_modes_dia_sys} shows that the first mode was the most important. The same was true for the thickness model. Figure \ref{fig:random_forest_feature_importance__100_shape_eigenvalues__and_100_thickness_modes_dia_sys} suggests that many of the first 15 modes are of relatively high importance. However, it is worth noting that the importance of particular shape and thickness modes increased when used along with the remaining features in Table \ref{table:featurelist} (e.g. shape mode 14 and thickness mode 7). This suggests that these modes provide different information to the ejection fraction and volumes measurements, and could represent an interesting avenue for future research.

Figure \ref{fig:classification_error__5kfold_LOO__all_72_dia_sys} shows the classification error when all of the features in Table \ref{table:featurelist} were used, showing that the final random forest classification was an improvement on the use of standard clinical parameters such as ejection fraction.

Shape and thickness variation from the remaining modes could be analysed and work could be done to investigate whether similar classification success could be achieved with a reduced number of features. 

 




\section{Acknowledgements}
This work has been supported by the Engineering and Physical Sciences Research Council (EPSRC) and the Medical Research Council (MRC).
% ---- Bibliography ----
\bibliographystyle{splncs03}
\bibliography{references}

\end{document}

